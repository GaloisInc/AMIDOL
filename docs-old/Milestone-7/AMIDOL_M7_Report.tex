\documentclass[11pt]{article}
\usepackage[parfill]{parskip}
\usepackage{graphicx}
\usepackage{wrapfig}
\usepackage{subcaption}
\usepackage[top=1in, bottom=1in, left=1in, right=1in]{geometry}
\bibliographystyle{plain}
\usepackage{amsmath}
\usepackage{amsfonts}
\usepackage{hyperref}
\usepackage{listings}
\usepackage{xcolor}

\usepackage{listings}
\usepackage{xcolor}
 
\definecolor{codegreen}{rgb}{0,0.6,0}
\definecolor{codegray}{rgb}{0.5,0.5,0.5}
\definecolor{codepurple}{rgb}{0.58,0,0.82}
\definecolor{backcolour}{rgb}{0.95,0.95,0.92}

\lstdefinestyle{mystyle}{
    backgroundcolor=\color{backcolour},   
    commentstyle=\color{codegreen},
    keywordstyle=\color{magenta},
    numberstyle=\tiny\color{codegray},
    stringstyle=\color{codepurple},
    basicstyle=\ttfamily\footnotesize,
    breakatwhitespace=false,         
    breaklines=true,                 
    captionpos=b,                    
    keepspaces=true,                 
    numbers=left,                    
    numbersep=5pt,                  
    showspaces=false,                
    showstringspaces=false,
    showtabs=false,                  
    tabsize=2
  }

 \lstdefinelanguage{Julia}%
  {morekeywords={abstract,break,case,catch,const,continue,do,else,elseif,%
      end,export,false,for,function,immutable,import,importall,if,in,%
      macro,module,otherwise,quote,return,switch,true,try,type,typealias,%
      using,while},%
   sensitive=true,%
   morecomment=[l]\#,%
   morecomment=[n]{\#=}{=\#},%
   morestring=[s]{"}{"},%
   morestring=[m]{'}{'},%
}[keywords,comments,strings]%
 
\lstset{style=mystyle}

%%%%%%%%%%%%%%%%%%%%%%%%%%%%%%%%%%%%%%%%%%%%%%%%%%%%%%%%%%%%%%%
\usepackage{fancyhdr}
\pagestyle{fancy}
%%% Please add the author's last names
\lhead{Galois TA2 AMIDOL}
\rhead{ASKE Milestone 6 Report}
%%% Please use \cfoot{} to remove page numbers
\cfoot{ }
\renewcommand{\headrulewidth}{0pt}
\renewcommand{\footrulewidth}{0pt}
%%%%%%%%%%%%%%%%%%%%%%%%%%%%%%%%%%%%%%%%%%%%%%%%%%%%%%%%%%%%%%&
\usepackage{amsthm}
 
\theoremstyle{definition}
\newtheorem{definition}{Definition}[section]

\colorlet{punct}{red!60!black}
\definecolor{background}{HTML}{EEEEEE}
\definecolor{delim}{RGB}{20,105,176}
\colorlet{numb}{magenta!60!black}

\usepackage{amsthm}

\newtheorem{mydef}{Definition}

\lstdefinelanguage{json}{
    basicstyle=\normalfont\ttfamily,
    numbers=left,
    numberstyle=\scriptsize,
    stepnumber=1,
    numbersep=8pt,
    showstringspaces=false,
    breaklines=true,
    frame=lines,
    backgroundcolor=\color{background},
    literate=
     *{0}{{{\color{numb}0}}}{1}
      {1}{{{\color{numb}1}}}{1}
      {2}{{{\color{numb}2}}}{1}
      {3}{{{\color{numb}3}}}{1}
      {4}{{{\color{numb}4}}}{1}
      {5}{{{\color{numb}5}}}{1}
      {6}{{{\color{numb}6}}}{1}
      {7}{{{\color{numb}7}}}{1}
      {8}{{{\color{numb}8}}}{1}
      {9}{{{\color{numb}9}}}{1}
      {:}{{{\color{punct}{:}}}}{1}
      {,}{{{\color{punct}{,}}}}{1}
      {\{}{{{\color{delim}{\{}}}}{1}
      {\}}{{{\color{delim}{\}}}}}{1}
      {[}{{{\color{delim}{[}}}}{1}
      {]}{{{\color{delim}{]}}}}{1},
}

\newcommand{\amidol}{\textsc{AMIDOL}}

\def\signed #1{{\leavevmode\unskip\nobreak\hfil\penalty50\hskip2em
  \hbox{}\nobreak\hfil(#1)%
  \parfillskip=0pt \finalhyphendemerits=0 \endgraf}}

\newsavebox\mybox
\newenvironment{aquote}[1]
  {\savebox\mybox{#1}\begin{quote}}
  {\signed{\usebox\mybox}\end{quote}}

\date{\vspace{-5ex}}
% Use this to get rid of the date

\usepackage{authblk}
\author[1]{Eric Davis}
\author[1]{Alec Theriault}
\author[1]{Ryan Wright}
\affil[1]{Galois, Inc}

%\setcounter{page}{0}



\title{May ASKE Milestone 7 Report for \amidol{}}

\begin{document}
\maketitle
\vspace{10pt}

\section{Introduction}

In this report we discuss recent extensions to \amidol{}'s framework
which seek to extend \amidol{}'s Abstract Knowledge Layer and
Formulation and Paletter generation capabilities, and briefly discuss
extensions to \amidol{}'s ability to work with additional intermediate
representations at the Structured Knowledge Layer, and support for
additional backends for its Executable Knowledge Layer.

\section{Formulations}

\amidol{}'s Abstract Knowledge Layer focuses on the representation,
manipulation, and extension of problem formulations as part of the
metamodeling process.

\begin{definition}{Formulation}
  A formulation is a high-level domain model which represents and
  encodes semantic domain knowledge about a complex system.
  Formulations are the model-stack equivalent of a high level
  language and should focus on being accessible to domain scientists.
  While formal semantics may be implied by a formulation, the primary
  point of formulating a model is not writing down executable code or
  mathematical representations.  It is
  writing down assumptions and knowledge which can, at a later stage,
  be used to infer these properties of a model.

  Formulations should be represented in a form that is close to
  natural language, or natural representations, such as diagrams, used
  by domain scientists.

  Formulations define the Abstract Knowledge Layer of the modeling stack.
\end{definition}

The primary mechanism used for formulations in \amidol{} are Visual
Domain Specific Ontological Languages (VDSOLs).  A VDSOL is defined by
a Formulation Palette,

\begin{definition}{Formulation Palette}
  A formulation palette is a set $P = \{ p_0, p_1, \ldots\}$ of
  palette elements.
\end{definition}

\begin{definition}{Palette Element}
  A palette element is a type which can be instantiated in a
  formulation as a concrete occurance of the element.  These instance
  elements must have unique names, and can be connected with arcs to
  define structured co-spans of instance elements which can be
  compiled into \amidol{}'s intermediate representation.

  A palette element is defined by the combination of a palette
  template $T$, and a definition in \amidol{}'s intermediate
  representation $R$ into the tuple $(T,R)$

  An instance element is a palette element, combined with a unique
  identifier $N$ and a set of parameters and constants $C$, $(T, R, N,
  C)$.
\end{definition}

\begin{definition}{Palette Template}
  A palette template is a category defined by an input set, output
  set, and measure set, over an unknown or undefined \amidol{} IR.
  Palette templates are essentially container categories for
  compatible \amidol{} IR models whose input and output cardinality
  and typing match, making them interchangable.
\end{definition}

A Formulation in \amidol{} is a set of palette element instances
composed together by a set of relations, $A \rightarrow B$ such that
for the output cardinality of $A$ and the input cardinality of $B$ are
the same, and have compatible types.  Type relations allow
restrictions on model composition which are richer than input/output
cardinality; such as that used by all current \amidol{} palettes which
states if $A$ is a noun, $B$ must be a verb, and if $A$ is a verb, $B$
must be a noun.  Further restrictions could be used to apply types to
individual inputs or outputs to restrict their composability to.

\subsection{Formulation Paradigm}

The process of model formulation in \amidol{} seeks to allow domain
experts to construct meta-models in a novel way, using VDSOLs.  These
VDSOLs utilize an
underlying intermediate abstract representation to give formal meaning
to the intuitive process diagrams scientists and domain experts
normally create.  The intention is to remove the burden of having to
code models explicitly, and enable domain experts to build models of
complex systems which are easier to maintain, validate, and verify,
and avoid common pitfalls of monolithic and hand-coded
implementations.

While to date these formulations have all been through the use of
diagrams, we are extending \amidol{} to support other types of formal
languages that are natural for certain domains.

\subsection{Formulation Languages}

To explore other types of visual languages and diagrams, we are
extending \amidol{} to support systems of stochastic chemical
equations via standard notations for networks of chemical reactions of
the form \[A +B \overset{\beta}{\rightarrow} C\] which is interpreted
as ``a molecule of $A$ combines with a molecule of $B$ to form a
molecule of $C$ with propensity $\beta$''.  We use the following
formal grammar to represent stochastic chemical equations in \amidol{}
as a formulation:

\lstinputlisting[caption=Stochastic Chemical Equation
Grammar]{equation.lark}

\amidol{} is capable of reading networks of chemical reactions, like
the following:

\begin{lstlisting}[caption=Example Stochastic Chemical Equation]
1.0,      A       -> B
2.0,      A + B <--> C + D
3.0, 4.0, A + B <--> C + D
\end{lstlisting}

and generating an abstract syntax tree like that shown in Figure
\ref{Fig:Reaction}.  This abstract syntax tree can then be used to
generate a model in the \amidol{} IR, representing each reactant with
a state variable, and constructing events from the equations.

\begin{figure}
  \includegraphics[width=\textwidth]{tree.png}
  \caption{Example Parse Tree from Stochastic Chemical Grammar}
  \label{Fig:Reaction}
\end{figure}

While designed to test the expressive capability of \amidol{}'s
Abstract Knowledge Layer in representing chemical equations, it can
also be used to represent other models, like the SIR model, as
follows:

\begin{lstlisting}[caption=Example Stochastic Chemical Equation]
beta / (S + I + R), S + I --> I
gamma,              I --> R
\end{lstlisting}

\section{Formulations from Models}

In order to support a joint demo with GTRI and the University of
Arizona, we have also been focusing on developing methods which allow
us to automatically generate formulations in a visual language from
models at the Structured Knowledge Layer.  To support this, we have
constructed a new module for \amidol{} which is able to translate
Julia abstract syntax trees into
\amidol{}'s IR.  These models are combined with grounding information
to perform the novel process of Formulation Inference in \amidol{}.

\subsection{Formulation Inference}

Formulation Inference is a formal procedure in \amidol{} for
automatically generating new palette elements from existing palette
templates, using an ontology of domain specific terms.

\begin{figure}
  \includegraphics[width=\textwidth]{H3N2vmap.png}
  \caption{Ontology search for viral template from grounding
    ``Influenza A virus subtype H3N2v''}
  \label{Fig:ontology1}
\end{figure}

Figure \ref{Fig:ontology1} shows a subset of the SNOMED CT ontology,
representing elements of the ontology with nodes labeled with their
SCTID.  In this example, the green node represents the term
``Influenza A virus subtype H3N2v'' while the red node represents the
term ``Virus (organism)''.  Formulation inference assumes the
existance of an annotated ontology, with nodes containing palette
templates as annotations.  In this example the ``Virus (organism)''
node has been annotated with the palette template for a viral
infection verb.

\section{The SNOMED Ontologies}

\section{Examples}

\begin{figure}
  \begin{center}
    \includegraphics[width=\textwidth]{H1N1.png}
  \end{center}
  \caption{Ontology search for viral template from grounding
    ``Influenza A (H1N1''}
\end{figure}

\begin{figure}
  \includegraphics[width=\textwidth]{H3N2map.png}
  \caption{Ontology search for viral template from grounding ``H3N2''}
\end{figure}

\begin{figure}
  \includegraphics[width=\textwidth]{Ebolamap.png}
  \caption{Ontology search for viral template from grounding ``Ebola''}
\end{figure}

  * Updated Diagram
  * Formulation vs. Model vs. Implementation
    * Formulation paradigm
    * Formulation languages
  * Julia AST -> IR
  * Snomed Ontology Summary
  * Palette Generation from Templates
    * Category Theory
    * Templating
    * Snomed and Palette Generation

\section{Other Recent AMIDOL Extensions}

\section{Resources, web sites, etc.}

The current \amidol{} source code, including example models and documentation, is available at the \amidol{} Github site \url{https://github.com/GaloisInc/AMIDOL}.

\bibliography{AMIDOL-MWS}

\end{document}
