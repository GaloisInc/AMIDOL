\documentclass[12pt]{galois-whitepaper}
\usepackage{listings}
\usepackage{float}
\usepackage{xspace}
\usepackage{color}
\usepackage{tikz}
\usepackage{url}
\usepackage{amsmath}
\usepackage{amscd}
\usepackage{verbatim}
\usepackage{fancyvrb}
\let\verbatiminput=\verbatimtabinput
\VerbatimFootnotes
\DefineVerbatimEnvironment{code}{Verbatim}{}
\DefineVerbatimEnvironment{pseudoCode}{Verbatim}{}
%\hyphenation{Saw-Script}
%\newcommand{\sawScript}{{\sc SawScript}\xspace}
\renewcommand{\textfraction}{0.05}
\renewcommand{\topfraction}{0.95}
\renewcommand{\bottomfraction}{0.95}
\renewcommand{\floatpagefraction}{0.35}
\setcounter{totalnumber}{5}
\definecolor{MyGray}{rgb}{0.9,0.9,0.9}
\makeatletter\newenvironment{graybox}{%
   \begin{lrbox}{\@tempboxa}\begin{minipage}{\columnwidth}}{\end{minipage}\end{lrbox}%
   \colorbox{MyGray}{\usebox{\@tempboxa}}
}\makeatother

\setlength{\parskip}{0.6em}
\setlength{\abovecaptionskip}{0.5em}

\lstset{
         basicstyle=\footnotesize\ttfamily, % Standardschrift
         %numbers=left,               % Ort der Zeilennummern
         numberstyle=\tiny,          % Stil der Zeilennummern
         %stepnumber=2,               % Abstand zwischen den Zeilennummern
         numbersep=5pt,              % Abstand der Nummern zum Text
         tabsize=2,                  % Groesse von Tabs
         extendedchars=true,         %
         breaklines=true,            % Zeilen werden Umgebrochen
         keywordstyle=\color{red},
                frame=lrtb,         % left, right, top, bottom frames.
 %        keywordstyle=[1]\textbf,    % Stil der Keywords
 %        keywordstyle=[2]\textbf,    %
 %        keywordstyle=[3]\textbf,    %
 %        keywordstyle=[4]\textbf,   \sqrt{\sqrt{}} %
         stringstyle=\color{white}\ttfamily, % Farbe der String
         showspaces=false,           % Leerzeichen anzeigen ?
         showtabs=false,             % Tabs anzeigen ?
         xleftmargin=10pt, % was 17
         xrightmargin=5pt,
         framexleftmargin=5pt, % was 17
         framexrightmargin=-1pt, % was 5pt
         framexbottommargin=4pt,
         %backgroundcolor=\color{lightgray},
         showstringspaces=false      % Leerzeichen in Strings anzeigen ?
}

\author{Eric Davis, Alec Theriault, and Ryan Wright}
\title{September ASKE Milestone 9 Report for AMIDOL}
\date{9/30/2019}
\begin{document}
\maketitle

\vspace*{2cm}
\tableofcontents

\section{Introduction}

Focus on project objectives

\begin{itemize}
  \item Graph-based knowledge representation utilizing meta-model
    ontology. Development of rich outcome representation in the VDSOL
    ontologies and meta-models to improve explainability of the model
    for practitioners.
  \item Verification and validation of inference engine. Verification
    and validation of back-end algorithms using benchmarks developed
    for 2.2 to find possible flaws in prototype
    implementation. Refinement of metrics developed in 1.5.
  \item Refinement of VDSOL, AFI, and Inference Engine. Based on
    lessons learned in task 2.1, and V\&V efforts for 2.4, we will
    update the VDSOL, AFI, and inference engine implementations for
    the scripted demonstration of the new system for use in a real
    world example of crises response to an emerging epidemic.
    
\end{itemize}

\section{Machine Generation of Hypotheses}

\section{Machine Evaluation of Hypotheses}

\section{Interpretation of Model Outputs}

\subsection{Characteristic Identification}

\subsection{Flaw Identification and Debugging}

\section{Ongoing Work}

\subsection{Design of Experiments}

\subsection{Treatment of Data}

\subsection{Metric Algebras}

\subsection{Formulations in Multiple VDSOLs/Across VDSOLs}


\end{document}
