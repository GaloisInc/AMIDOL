\documentclass[11pt]{article}
\usepackage[parfill]{parskip}
\usepackage{graphicx}
\usepackage{wrapfig}
\usepackage{subcaption}
\usepackage[top=1in, bottom=1in, left=1in, right=1in]{geometry}
\bibliographystyle{plain}
\usepackage{amsmath}
\usepackage{amsfonts}
\usepackage{hyperref}
\usepackage[T1]{fontenc}
%%%%%%%%%%%%%%%%%%%%%%%%%%%%%%%%%%%%%%%%%%%%%%%%%%%%%%%%%%%%%%%
\usepackage{fancyhdr}
\pagestyle{fancy}
%%% Please add the author's last names
\lhead{Davis, Theriault, Orhai, Westbrook, and Wright}
\rhead{Modeling the World's Systems, 2019}
%%% Please use \cfoot{} to remove page numbers
\cfoot{ }
\renewcommand{\headrulewidth}{0pt}
\renewcommand{\footrulewidth}{0pt}
%%%%%%%%%%%%%%%%%%%%%%%%%%%%%%%%%%%%%%%%%%%%%%%%%%%%%%%%%%%%%%&
\usepackage{titlesec}
\titlespacing{\section}{0pt}{\parskip}{-.5\parskip}
\titlespacing{\subsection}{0pt}{\parskip}{- .5\parskip}
\titlespacing{\subsubsection}{0pt}{\parskip}{- .5\parskip}
\newcommand{\closeup}{\setlength{\itemsep}{-4pt}}

\newcommand{\amidol}{\textsc{AMIDOL}}

\def\signed #1{{\leavevmode\unskip\nobreak\hfil\penalty50\hskip2em
  \hbox{}\nobreak\hfil(#1)%
  \parfillskip=0pt \finalhyphendemerits=0 \endgraf}}

\newsavebox\mybox
\newenvironment{aquote}[1]
  {\savebox\mybox{#1}\begin{quote}}
  {\signed{\usebox\mybox}\end{quote}}

\date{\vspace{-5ex}}
% Use this to get rid of the date

\usepackage{authblk}
\author{Eric Davis}
\author{Alec Theriault}
\author{Ryan Wright}
\affil{Galois, Inc}

%\setcounter{page}{0}



\title{Proof Sketches for the \amidol{} Intermediate Representation}

\begin{document}
\maketitle
\vspace{10pt}
\begin{abstract}
\end{abstract}

\section{Introduction}

One instruction set computer (OISC) and ultimate reduced instruction set computer (URISC) \cite{mavaddat1988urisc}.

\section{Acknowledgments}

This research has been supported by DARPA contract DARPA-PA-18-02-AIE-FP-039.

\section{Resources, web sites, etc.}

The current \amidol{} source code, including example models and documentation, is available at the \amidol{} Github site \url{https://github.com/GaloisInc/AMIDOL}.

Three classes of Petri nets:
\begin{itemize}
\item Condition-event nets (C/E-nets).
\item Place-transition nets (P/T-nets).
\item Predicate-event nets (P/E-nets).
\end{itemize}
\cite{reisig2012petri}

To be classed as a P/T-net the Petri net must lift the restriction of a single token residing in a given place, which is present for C/E-nets.  While this opens the door to infinite state-spaces, it also gives a petri-net the equivalent of an infinite bi-directional tape required for Turing machines.  

\cite{ciardo1994petri}


%\newcommand*{\doi}[1]{\href{http://dx.doi.org/#1}{doi: #1}}
\bibliography{AMIDOL-MWS}



\end{document}
